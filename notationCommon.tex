\usepackage{amsmath,amssymb}
\usepackage{mathtools}
\usepackage{etoolbox}

% the following line is in imports.tex
%\mathtoolsset{showonlyrefs=true}
%\usepackage{filecontents}


% works exactly as pragmaonce in C
% https://tex.stackexchange.com/questions/195157/is-there-any-analog-to-pragma-once-in-latex
\makeatletter
\let\pragma@iinput=\@iinput
\def\@iinput#1{\xdef\@pragmafile{#1}\pragma@iinput{#1}}
\def\@pragmafile{default}
\def\pragmaonce{%
	\csname pragma@\@pragmafile\endcsname
	\global\expandafter\let \csname pragma@\@pragmafile\endcsname = \endinput
}
\makeatother


%%%%%%%%%%%%%%%%%% generic notations
\DeclareMathOperator*{\argmin}{arg\,min}
\DeclareMathOperator*{\argmax}{arg\,max}
\DeclareMathOperator*{\maximize}{maximize}
\DeclareMathOperator*{\sign}{sgn}

\ifdefined\ceil
\else
\DeclarePairedDelimiter\ceil{\lceil}{\rceil}
\DeclarePairedDelimiter\floor{\lfloor}{\rfloor}
\fi
\DeclarePairedDelimiterX{\inprd}[2]{\langle}{\rangle}{#1, #2}
\newcommand{\inprdS}[2]{\left\langle#1,#2\right\rangle}


\newcommand{\ellnrm}[1]{\ell_{#1}}
\newcommand{\LnrmS}[1]{\left\lVert #1 \right\rVert}
\newcommand{\Lnrm}[1]{\lVert #1 \rVert}
\newcommand{\Lzro}[1]{\lVert #1 \rVert_0}
\newcommand{\Lone}[1]{\lVert #1 \rVert_1}
\newcommand{\Ltwo}[1]{\lVert #1 \rVert_2}
\newcommand{\Linf}[1]{\lVert #1 \rVert_\infty}
\newcommand{\wt}{\text{wt}}

\newcommand{\EV}{{\mathbb{E}}}
\newcommand{\EVS}[1]{{\mathbb{E}\left[#1\right]}}
\newcommand{\Var}{\mathbb{V}}
\newcommand{\VarLong}{\text{Var}}
\newcommand{\Rel}{\mathbb{R}}
\newcommand{\nball}{B_\epsilon}

\newcommand{\Hb}{H_b}
\newcommand{\arnn}[1]{\xrightarrow{\text{#1}}}
\newcommand{\nballex}{\nball_{\infty, \epsilon}}
\newcommand{\bigO}{\mathcal{O}}
\newcommand{\indctr}{\mathbbm{1}}
\newcommand{\allones}{\mathbf{1}}
\newcommand{\allzeros}{\mathbf{0}}
\newcommand{\iid}{i.i.d.}
%https://tex.stackexchange.com/questions/74125/how-do-i-put-text-over-symbols
\newcommand{\defn}{\mathrel{\overset{\makebox[0pt]{\mbox{\normalfont\tiny\sffamily def}}}{=}}}
%\newcommand{\defn}{{\triangleq}}%\coloneqq
\newcommand{\lce}{\text{l.c.e}}
\newcommand{\convexhull}{\text{conv}}
\newcommand{\st}{\text{s.t.}}
\newcommand{\Prob}{\text{P}}



%%%%%%%%%%%%%%%%%% caliographic chars
\newcommand{\cc}[1]{\mathcal{#1}}
\newcommand{\calhX}{\hat{\cc{X}}}
\newcommand{\calhY}{\hat{\cc{Y}}}
\newcommand{\calX}{\cc{X}}
\newcommand{\calY}{\cc{Y}}
\newcommand{\calC}{\cc{C}}
\newcommand{\calD}{\cc{D}}
\newcommand{\calA}{\cc{A}}
\newcommand{\calU}{\cc{U}}
\newcommand{\calV}{\cc{V}}
\newcommand{\calE}{\cc{E}}
\newcommand{\calM}{\cc{M}}
\newcommand{\calG}{\cc{G}}
\newcommand{\calT}{\cc{T}}
\newcommand{\calS}{\cc{S}}
\newcommand{\calB}{\cc{B}}
\newcommand{\calW}{\cc{W}}
\newcommand{\calN}{\cc{N}}
\newcommand{\calF}{\cc{F}}
\newcommand{\calR}{\cc{R}}
\newcommand{\calL}{\cc{L}}
\newcommand{\calJ}{\cc{J}}
\newcommand{\calK}{\cc{K}}


\newcommand{\GFt}{\GF{2}}
\newcommand{\GF}[1]{\mathbb{F}_{#1}}




%%%%%%%%%%%%%%%%%% text handling
\newcommand{\emphtwo}[1]{{\textit{#1}}}
\newcommand{\emphone}[1]{{\textbf{#1}}}
\newcommand{\Cred}[1]{{\color{red}{#1}}}
\definecolor{darkgreen}{rgb}{0.0, 0.5, 0.0}
\newcommand{\Cgrn}[1]{{\color{darkgreen}{#1}}}
\newcommand{\Cblue}[1]{{\color{blue}{#1}}}
\newcommand{\bluetria}{\textcolor[HTML]{00008b}{$\blacktriangleright~$}}    % beamer bullet

%%%%%%%%%%%%%%%%%% strike out for both text and math mode
% https://tex.stackexchange.com/questions/20609/strikeout-in-math-mode
\newcommand{\stkout}[1]{\ifmmode\text{\sout{\ensuremath{#1}}}\else\sout{#1}\fi}




%%%%%%%%%%%%%%%%%% sections/counter handling
\newcommand{\fig}[1]{{Fig.~\ref{fig:#1}}}
\newcommand{\tbl}[1]{{Table~\ref{tbl:#1}}}
\newcommand{\secn}[1]{{Sec.~\ref{secn:#1}}}
\newcommand{\Secn}[1]{{Section~\ref{secn:#1}}}
\newcommand{\apdx}[1]{{App.~\ref{secn:#1}}}
\newcommand{\Apdx}[1]{{Appendix~\ref{secn:#1}}}
\newcommand{\chap}[1]{{Chapter~\ref{secn:#1}}}
\newcommand{\deftn}[1]{{Definition~\ref{deftn:#1}}}
\newcommand{\thrm}[1]{{Theorem~\ref{thrm:#1}}}
\newcommand{\colly}[1]{{Corollary~\ref{colly:#1}}}
\newcommand{\lmma}[1]{{Lemma~\ref{lmma:#1}}}
\newcommand{\algo}[1]{{Alg.~\ref{algo:#1}}}
\newcommand{\Algo}[1]{{Algorithm~\ref{algo:#1}}}
% \newcommand{\eqn}[1]{{(\ref{eqn:#1})}}
% https://tex.stackexchange.com/questions/196446/how-to-number-only-labeled-equations
\newcommand{\eqn}[1]{{\eqref{eqn:#1}}}
\newcommand{\citepage}[1]{{{p.~#1}}}

\newcommand{\bullettopic}[1]{{\noindent\emph{$\bullet$ #1}}}
\newcommand{\nobullettopic}[1]{{\noindent\emph{#1}}}
\newcommand{\boldtopic}[1]{~\\\noindent{\bf{#1}}}


%%%%%%%%%%%%%%%%%%  counters for theorems/lemma/definitions
\newcommand\numberthis{\addtocounter{equation}{1}\tag{\theequation}}


%%%%%%%%%%%%%%%%%%  check-if-counter-exists
% https://tex.stackexchange.com/questions/155776/check-if-counter-exists
% \ifcounter{chapter}{exists}{doesn't exist}
\makeatletter
\newcommand*\ifcounter[1]{%
	\ifcsname c@#1\endcsname
	\expandafter\@firstoftwo
	\else
	\expandafter\@secondoftwo
	\fi
}
\makeatother



%%%%%%%%%%%%%%%%%% following are already defined for beamer template
% https://tex.stackexchange.com/questions/41496/how-to-define-an-environment-only-if-it-is-not-defined-yet-using-etoolbox
\ifcsmacro{theorem}{}{

	\ifcounter{chapter}{
		\def\theoremresetcounter{chapter}
	}{
		\def\theoremresetcounter{section}
	}
	
	\newtheorem{theorem}{Theorem}%[\theoremresetcounter]
	\newtheorem{definition}{Definition}%[\theoremresetcounter]
	\newtheorem{lemma}{Lemma}%[\theoremresetcounter]
	\newtheorem{corollary}{Corollary}%[\theoremresetcounter]
	\newtheorem{assumption}{Assumption}%[\theoremresetcounter]
	
	\newcounter{example}[section]
	\newenvironment{example}[1][]{\refstepcounter{example}\par\medskip
		\textbf{Example~\thesection.\theexample. #1} \rmfamily}{\medskip}
	\newcounter{problem}[section]
	\newenvironment{problem}[1][]{\refstepcounter{problem}\par\medskip
		\textbf{Problem~\thesection.\theproblem. #1} \rmfamily}{\medskip}
}


\def\enumtheoremstart{\begin{enumerate}[noitemsep,label=(\roman*)]}
	\def\enumtheoremend{\end{enumerate}}
\newcommand\enumtheorem[1]{\enumtheoremstart#1\enumtheoremend}
